% !TeX program = xelatex

\documentclass[12pt]{article}

\usepackage{ctex}
\usepackage[a6paper]{geometry}

\begin{document}

\title{统编语文1-6年级112首古诗词}
\author{}
\date{}

\maketitle

% \tableofcontents

\centering

\section*{一年级上册}

\subsection*{01 咏鹅}

【唐】骆宾王

鹅,鹅,鹅,曲项向天歌。

白毛浮绿水,红掌拨清波。

\subsection*{02 江南}

汉乐府

江南可采莲,莲叶何田田。

鱼戏莲叶间。

鱼戏莲叶东,鱼戏莲叶西,

鱼戏莲叶南,鱼戏莲叶北。

\subsection*{03 画}

远看山有色,近听水无声。

春去花还在,人来鸟不惊。

\subsection*{04 悯农(其二)}

【唐】李绅

锄禾日当午,汗滴禾下土。

谁知盘中餐,粒粒皆辛苦。

\subsection*{05 古朗月行(节选)}

【唐】李白

小时不识月,呼作白玉盘。

又疑瑶台镜,飞在青云端。

\subsection*{06 风}

【唐】李峤

解落三秋叶,能开二月花。

过江千尺浪,入竹万竿斜。

\newpage

\section*{一年级下册}

\subsection*{01 春晓}

【唐】 孟浩然

春眠不觉晓,处处闻啼鸟。

夜来风雨声,花落知多少。

\subsection*{02 赠汪伦}

【唐】李白

李白乘舟将欲行,忽闻岸上踏歌声。

桃花潭水深千尺,不及汪伦送我情。

\subsection*{03 静夜思}

【唐】李白

床前明月光,疑是地上霜。

举头望明月,低头思故乡。

\subsection*{04 寻隐者不遇}

【唐】贾岛 

松下问童子,言师采药去。

只在此山中,云深不知处。


\subsection*{05 池上}

【唐】白居易


小娃撑小艇,偷采白莲回。

不解藏踪迹,浮萍一道开。

\subsection*{06 小池}

【宋】杨万里

泉眼无声惜细流,树阴照水爱晴柔。

小荷才露尖尖角,早有蜻蜓立上头。

\subsection*{07 画鸡}

【明】唐寅

头上红冠不用裁,满身雪白走将来。

平生不敢轻言语,一叫千门万户开。

\newpage

\section*{二年级上册}

\subsection*{01 梅花}

【宋】王安石

墙角数枝梅,凌寒独自开。

遥知不是雪,为有暗香来。

\subsection*{02 小儿垂钓}

【唐】胡令能

蓬头稚子学垂纶,侧坐莓苔草映身。

路人借问遥招手,怕得鱼惊不应人。

\subsection*{03 登鹳雀楼}

【唐】王之涣

白日依山尽,黄河入海流。

欲穷千里目,更上一层楼。

\subsection*{04 望庐山瀑布}

【唐】李白

日照香炉生紫烟,遥看瀑布挂前川。

飞流直下三千尺,疑是银河落九天。

\subsection*{05 江雪}

【唐】柳宗元

千山鸟飞绝,万径人踪灭。

孤舟蓑笠翁,独钓寒江雪。

\subsection*{06 夜宿山寺}

【唐】李白

危楼高百尺,手可摘星辰。

不敢高声语,恐惊天上人。

\subsection*{07 敕勒歌}

北朝民歌

敕勒川,阴山下,

天似穹庐,笼盖四野。

天苍苍,野茫茫,

风吹草低见牛羊。

\newpage

\section*{二年级下册}

\subsection*{01 村居}

【清】高鼎

草长莺飞二月天,拂堤杨柳醉春烟。

儿童散学归来早,忙趁东风放纸鸢。

\subsection*{02 咏柳}

【唐】贺知章

碧玉妆成一树高,万条垂下绿丝绦。

不知细叶谁裁出,二月春风似剪刀。

\subsection*{03 赋得古原草送别(节选)}

【唐】白居易

离离原上草,一岁一枯荣。

野火烧不尽,春风吹又生。

\subsection*{04 晓出净慈寺送林子方}

【宋】杨万里

毕竟西湖六月中,风光不与四时同。

接天莲叶无穷碧,映日荷花别样红。

\subsection*{05 绝句}

【唐】杜甫

两个黄鹂鸣翠柳,一行白鹭上青天。

窗含西岭千秋雪,门泊东吴万里船。

\subsection*{06 悯农(其一)}

【唐】李绅

春种一粒粟,秋收万颗子。

四海无闲田,农夫犹饿死。

\subsection*{07 舟夜书所见}

【清】查慎行

月黑见渔灯,孤光一点萤。

微微风簇浪,散作满河星。

\newpage

\section*{三年级上册}

\subsection*{01 所见}

【清】袁枚

牧童骑黄牛,歌声振林樾。

意欲捕鸣蝉,忽然闭口立。

\subsection*{02 山行}

【唐】杜牧

远上寒山石径斜,白云生处有人家。

停车坐爱枫林晚,霜叶红于二月花。

\subsection*{03 赠刘景文}

【宋】苏轼

荷尽已无擎雨盖,菊残犹有傲霜枝。

一年好景君须记,最是橙黄橘绿时。

\subsection*{04 夜书所见}

【宋】叶绍翁

萧萧梧叶送寒声,江上秋风动客情。

知有儿童挑促织,夜深篱落一灯明。

\subsection*{05 望天门山}

【唐】李白

天门中断楚江开,碧水东流至此回。

两岸青山相对出,孤帆一片日边来。

\subsection*{06 饮湖上初晴后雨}

【宋】苏轼

水光潋滟晴方好,山色空蒙雨亦奇。

欲把西湖比西子,淡妆浓抹总相宜。

\subsection*{07 望洞庭}

【唐】刘禹锡

湖光秋月两相和,潭面无风镜未磨。

遥望洞庭山水翠,白银盘里一青螺。

\subsection*{08 早发白帝城}

【唐】李白

朝辞白帝彩云间,千里江陵一日还。

两岸猿声啼不住,轻舟已过万重山。

\subsection*{09 采莲曲}

【唐】王昌龄

荷叶罗裙一色裁,芙蓉向脸两边开。

乱入池中看不见,闻歌始觉有人来。

\newpage

\section*{三年级下册}

\subsection*{01 绝句}

【唐】杜甫

迟日江山丽,春风花草香。

泥融飞燕子,沙暖睡鸳鸯。

\subsection*{02 惠崇春江晚景}

【宋】苏轼

竹外桃花三两枝,春江水暖鸭先知。

蒌蒿满地芦芽短,正是河豚欲上时。

\subsection*{03 三衢道中}

【宋】曾几

梅子黄时日日晴,小溪泛尽却山行。

绿阴不减来时路,添得黄鹂四五声。

\subsection*{04 忆江南}

【唐】白居易

江南好,风景旧曾谙。

日出江花红胜火,春来江水绿如蓝。

能不忆江南?

\subsection*{05 元日}

【宋】王安石

爆竹声中一岁除,春风送暖入屠苏。

千门万户曈曈日,总把新桃换旧符。

\subsection*{06 清明}

【唐】杜牧

清明时节雨纷纷,路上行人欲断魂。

借问酒家何处有?牧童遥指杏花村。

\subsection*{07 九月九日忆山东兄弟}

【唐】王维

独在异乡为异客,每逢佳节倍思亲。

遥知兄弟登高处,遍插茱萸少一人。

\subsection*{08 滁州西涧}

【唐】韦应物

独怜幽草涧边生,上有黄鹂深树鸣。

春潮带雨晚来急,野渡无人舟自横。

\subsection*{09 大林寺桃花}

【唐】白居易

人间四月芳菲尽,山寺桃花始盛开。

长恨春归无觅处,不知转入此中来。

\newpage

\section*{四年级上册}

\subsection*{01 鹿柴}

【唐】王维

空山不见人,但闻人语响。

返景入深林,复照青苔上。

\subsection*{02 暮江吟}

【唐】白居易

一道残阳铺水中,半江瑟瑟半江红。

可怜九月初三夜,露似真珠月似弓。

\subsection*{03 题西林壁}

【宋】苏轼

横看成岭侧成峰,远近高低各不同。

不识庐山真面目,只缘身在此山中。

\subsection*{04 雪梅}

【宋】卢钺

梅雪争春未肯降,骚人搁笔费评章。

梅须逊雪三分白,雪却输梅一段香。

\subsection*{05 嫦娥}

【唐】李商隐

云母屏风烛影深,长河渐落晓星沉。

嫦娥应悔偷灵药,碧海青天夜夜心。

\subsection*{06 出塞}

【唐】王昌龄

秦时明月汉时关,万里长征人未还。

但使龙城飞将在,不教胡马度阴山。

\subsection*{07 凉州词}

【唐】王翰

葡萄美酒夜光杯,欲饮琵琶马上催。

醉卧沙场君莫笑,古来征战几人回?

\subsection*{08 夏日绝句}

【宋】李清照

生当作人杰,死亦为鬼雄。

至今思项羽,不肯过江东。

\subsection*{09 别董大}

【唐】高适

千里黄云白日曛,北风吹雁雪纷纷。

莫愁前路无知己,天下谁人不识君?

\newpage

\section*{四年级下册}

\subsection*{01 宿新市徐公店}

【宋】杨万里

篱落疏疏一径深,树头新绿未成阴。

儿童急走追黄蝶,飞入菜花无处寻。

\subsection*{02 四时田园杂兴(其二十五)}

【宋】范成大

梅子金黄杏子肥,麦花雪白菜花稀。

日长篱落无人过,惟有蜻蜓蛱蝶飞。

\subsection*{03 清平乐·村居}

【宋】辛弃疾

茅檐低小,溪上青青草。

醉里吴音相媚好,白发谁家翁媪? 

大儿锄豆溪东,中儿正织鸡笼。

最喜小儿亡赖,溪头卧剥莲蓬。

\subsection*{04 卜算子·咏梅}

毛泽东

风雨送春归,飞雪迎春到。

已是悬崖百丈冰,犹有花枝俏。

俏也不争春,只把春来报。

待到山花烂漫时,她在丛中笑。

\subsection*{05 蜂}

【唐】罗隐

不论平地与山尖,无限风光尽被占。

采得百花成蜜后,为谁辛苦为谁甜?

\subsection*{06 独坐敬亭山}

【唐】李白

众鸟高飞尽,孤云独去闲。

相看两不厌,只有敬亭山。

\subsection*{07 芙蓉楼送辛渐}

【唐】王昌龄

寒雨连江夜入吴,平明送客楚山孤。

洛阳亲友如相问,一片冰心在玉壶。

\subsection*{08 塞下曲}

【唐】卢纶

月黑雁飞高,单于夜遁逃。

欲将轻骑逐,大雪满弓刀。

\subsection*{09 墨梅}

【元】王冕

我家洗砚池头树,朵朵花开淡墨痕。

不要人夸好颜色,只留清气满乾坤。

\newpage

\section*{五年级上册}

\subsection*{01 蝉}

【唐】虞世南

垂緌饮清露,流响出疏桐。

居高声自远,非是藉秋风。

\subsection*{02 乞巧}


【唐】林杰

七夕今宵看碧霄,牵牛织女渡河桥。

家家乞巧望秋月,穿尽红丝几万条。

\subsection*{03 示儿}

【宋】陆游

死去元知万事空,但悲不见九州同。

王师北定中原日,家祭无忘告乃翁。

\subsection*{04 题临安邸}

【宋】林升

山外青山楼外楼,西湖歌舞几时休?

暖风熏得游人醉,直把杭州作汴州。

\subsection*{05 己亥杂诗}

【清】龚自珍

九州生气恃风雷,万马齐喑究可哀。

我劝天公重抖擞,不拘一格降人才。

\subsection*{06 山居秋暝}

【唐】王维

空山新雨后,天气晚来秋。

明月松间照,清泉石上流。

竹喧归浣女,莲动下渔舟。

随意春芳歇,王孙自可留。

\subsection*{07 枫桥夜泊}

【唐】张继

月落乌啼霜满天,江枫渔火对愁眠。

姑苏城外寒山寺,夜半钟声到客船。

\subsection*{08 长相思}

【清】纳兰性德

山一程,水一程,身向榆关那畔行,夜深千帐灯。

风一更,雪一更,聒碎乡心梦不成,故园无此声。

\subsection*{09 渔歌子}

【唐】张志和

西塞山前白鹭飞,桃花流水鳜鱼肥。

青箬笠,绿蓑衣,斜风细雨不须归。

\subsection*{10 观书有感(其一)}

【宋】朱熹

半亩方塘一鉴开,天光云影共徘徊。

问渠那得清如许?为有源头活水来。

\subsection*{11 观书有感(其二)}

【宋】朱熹

昨夜江边春水生,蒙冲巨舰一毛轻。

向来枉费推移力,此日中流自在行。

\newpage

\section*{五年级下册}

\subsection*{01 四时田园杂兴(其三十一)}

【宋】范成大

昼出耘田夜绩麻,村庄儿女各当家。

童孙未解供耕织,也傍桑阴学种瓜。

\subsection*{02 稚子弄冰}

【宋】杨万里

稚子金盆脱晓冰,彩丝穿取当银铮。

敲成玉磬穿林响,忽作玻璃碎地声。

\subsection*{03 村晚}


【宋】雷震

草满池塘水满陂,山衔落日浸寒漪。

牧童归去横牛背,短笛无腔信口吹。

\subsection*{04 鸟鸣涧}

【唐】王维

人闲桂花落,夜静春山空。

月出惊山鸟,时鸣春涧中。

\subsection*{05 凉州词}

【唐】王之涣

黄河远上白云间,一片孤城万仞山。

羌笛何须怨杨柳,春风不度玉门关。

\subsection*{06 送元二使安西}

【唐】王维

渭城朝雨浥轻尘,客舍青青柳色新。

劝君更尽一杯酒,西出阳关无故人。

\subsection*{07 秋夜将晓出篱门迎凉有感}

【宋】陆游

三万里河东入海,五千仞岳上摩天。

遗民泪尽胡尘里,南望王师又一年。

\subsection*{08 寒菊}

【宋】郑思肖

花开不并百花丛,独立疏篱趣未穷。

宁可枝头抱香死,何曾吹落北风中。

\subsection*{09 乡村四月}

【宋】翁卷

绿遍山原白满川,子规声里雨如烟。

乡村四月闲人少,才了蚕桑又插田。

\newpage

\section*{六年级上册}

\subsection*{01 宿建德江}

【唐】孟浩然

移舟泊烟渚,日暮客愁新。

野旷天低树,江清月近人。

\subsection*{02 六月二十七日望湖楼醉书}

【宋】苏轼

黑云翻墨未遮山,白雨跳珠乱入船。

卷地风来忽吹散,望湖楼下水如天。

\subsection*{03 西江月·夜行黄沙道中}

【宋】辛弃疾

明月别枝惊鹊,清风半夜鸣蝉。

稻花香里说丰年,听取蛙声一片。

七八个星天外,两三点雨山前。

旧时茅店社林边,路转溪桥忽见。

\subsection*{04 过故人庄}

【唐】孟浩然
   
故人具鸡黍,邀我至田家。

绿树村边合,青山郭外斜。

开轩面场圃,把酒话桑麻。

待到重阳日,还来就菊花。

\subsection*{05 七律·长征}

毛泽东

红军不怕远征难,万水千山只等闲。

五岭逶迤腾细浪,乌蒙磅礴走泥丸。

金沙水拍云崖暖,大渡桥横铁索寒。

更喜岷山千里雪,三军过后尽开颜。

\subsection*{06 菩萨蛮·大柏地}

毛泽东

赤橙黄绿青蓝紫,谁持彩练当空舞?

雨后复斜阳,关山阵阵苍。

当年鏖战急,弹洞前村壁。

装点此关山,今朝更好看。

\subsection*{07 春日}

【宋】朱熹

胜日寻芳泗水滨,无边光景一时新。

等闲识得东风面,万紫千红总是春。

\subsection*{08 回乡偶书}

【唐】贺知章

少小离家老大回,乡音无改鬓毛衰。

儿童相见不相识,笑问客从何处来。

\subsection*{09 浪淘沙(其一)}

【唐】刘禹锡

九曲黄河万里沙,浪淘风簸自天涯。

如今直上银河去,同到牵牛织女家。

\subsection*{10 江南春}

【唐】杜牧

千里莺啼绿映红,水村山郭酒旗风。

南朝四百八十寺,多少楼台烟雨中。

\subsection*{11 书湖阴先生壁}

【宋】王安石

茅檐长扫净无苔,花木成畦手自栽。

一水护田将绿绕,两山排闼送青来。

\newpage

\section*{六年级下册}

\subsection*{01 寒食}

【唐】韩翃

春城无处不飞花,寒食东风御柳斜。

日暮汉宫传蜡烛,轻烟散入五侯家。

\subsection*{02 迢迢牵牛星}

迢迢牵牛星,皎皎河汉女。

纤纤擢素手,札札弄机杼。

终日不成章,泣涕零如雨。

河汉清且浅,相去复几许。

盈盈一水间,脉脉不得语。

\subsection*{03 十五夜望月}

【唐】王建

中庭地白树栖鸦,冷露无声湿桂花。

今夜月明人尽望,不知秋思落谁家?

\subsection*{04 长歌行}

汉乐府

青青园中葵,朝露待日晞。

阳春布德泽,万物生光辉。

常恐秋节至,焜黄华叶衰。

百川东到海,何时复西归?

少壮不努力,老大徒伤悲!

\subsection*{05 马诗}

【唐】李贺

大漠沙如雪,燕山月似钩。

何当金络脑,快走踏清秋。

\subsection*{06 石灰吟}

【明】于谦

千锤万凿出深山,烈火焚烧若等闲。

粉身碎骨浑不怕,要留清白在人间。

\subsection*{07 竹石}

【清】郑燮

咬定青山不放松,立根原在破岩中。

千磨万击还坚劲,任尔东西南北风。

\subsection*{08 采薇(节选)}

《诗经·小雅》

昔我往矣,杨柳依依。

今我来思,雨雪霏霏。

行道迟迟,载渴载饥。

我心伤悲,莫知我哀!

\subsection*{09 春夜喜雨}

【唐】杜甫

好雨知时节,当春乃发生。

随风潜入夜,润物细无声。

野径云俱黑,江船火独明。

晓看红湿处,花重锦官城。

\subsection*{10 闻官军收河南河北}

【唐】杜甫

剑外忽传收蓟北,初闻涕泪满衣裳。

却看妻子愁何在,漫卷诗书喜欲狂。

白日放歌须纵酒,青春作伴好还乡。

即从巴峡穿巫峡,便下襄阳向洛阳。

\subsection*{11 早春呈水部张十八员外}

【唐】韩愈

天街小雨润如酥,草色遥看近却无。

最是一年春好处,绝胜烟柳满皇都。

\subsection*{12 江上渔者}

【宋】范仲淹

江上往来人,但爱鲈鱼美。

君看一叶舟,出没风波里。

\subsection*{13 泊船瓜洲}

【宋】王安石

京口瓜洲一水间,钟山只隔数重山。

春风又绿江南岸,明月何时照我还。

\subsection*{14 游园不值}

【宋】叶绍翁

应怜屐齿印苍苔,小扣柴扉久不开。

春色满园关不住,一枝红杏出墙来。

\subsection*{15 卜算子·送鲍浩然之浙东}

【宋】王观

水是眼波横,山是眉峰聚。

欲问行人去那边?眉眼盈盈处。

才始送春归,又送君归去。

若到江南赶上春,千万和春住。

\subsection*{16 浣溪沙}

【宋】苏轼

游蕲水清泉寺,

寺临兰溪,溪水西流。

山下兰芽短浸溪,

松间沙路净无泥,

潇潇暮雨子规啼。

谁道人生无再少?

门前流水尚能西!

休将白发唱黄鸡。

\subsection*{17 永平乐}

【宋】黄庭坚

春归何处?寂寞无行路。

若有人知春去处,唤取归来同住。

春无踪迹谁知?除非问取黄鹂。

百啭无人能解,因风飞过蔷薇。


\end{document}
